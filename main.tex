\documentclass[prb, aps, twocolumn, centering, 12pt]{revtex4-2}

%-------------------------------
% Encoding & Language
%-------------------------------
\usepackage[T1]{fontenc}
\usepackage[english]{babel}
\usepackage[dvipsnames]{xcolor} % Color support

%-------------------------------
% Fonts & Math
%-------------------------------
\usepackage{eulervm}        % Euler fonts
\usepackage{amsmath, amssymb, mathtools, dsfont, nicefrac}
\usepackage{gensymb}        % Degree symbol
\usepackage{bm}              % Bold math symbols
\usepackage{braket}          % Bra-ket notation
\renewcommand{\mathbf}{\mathbold} % Optional bold math override
\renewcommand{\baselinestretch}{1}

%-------------------------------
% Graphics & Figures
%-------------------------------
\usepackage{graphicx}
\usepackage{tikz}
\usetikzlibrary{
    shapes.arrows,
    shapes.geometric,
    shapes.symbols,
    arrows.meta,
    calc,
    positioning,
    decorations.pathreplacing
}
\usepgflibrary{shadings}    % Shadings for TikZ
\usepackage{circuitikz}     % Circuits
\usepackage{float}           % Float positioning
\usepackage{caption}         % Figure captions
\captionsetup[figure]{font=footnotesize}

%-------------------------------
% Tables & Arrays
%-------------------------------
\usepackage{array}
\usepackage{dcolumn}

%-------------------------------
% Layout & Formatting
%-------------------------------
\usepackage[%
    text={7.25in,10in}
]{geometry}
\setlength{\columnsep}{0.5cm}
\usepackage{indentfirst}     % Indent first paragraph

%-------------------------------
% Color, Highlighting & Boxes
%-------------------------------
\usepackage{soul}            % Highlighting
\usepackage{color}
\usepackage[most]{tcolorbox} % Colored boxes
\usepackage{bbm}             % Blackboard math symbols
\usepackage{blindtext, lipsum} % Dummy text

%-------------------------------
% Headers & Footers
%-------------------------------
\usepackage{fancyhdr}
\pagestyle{fancy}
\fancyhf{}
% \rhead{PHYS 598, Liam Miller\qquad \thepage}

%-------------------------------
% Titles
%-------------------------------
\usepackage{titlesec}
% \titlespacing{\section}{0pt}{12pt plus 4pt minus 2pt}{4pt plus 2pt minus 2pt}
% \titlespacing{\subsection}{0pt}{12pt plus 4pt minus 2pt}{4pt plus 2pt minus 2pt}
% \titlespacing{\subsubsection}{0pt}{12pt plus 4pt minus 2pt}{4pt plus 2pt minus 2pt}

%-------------------------------
% Hyperlinks
%-------------------------------
\usepackage{hyperref}
% \hypersetup{pdfborder = {0 0 2}}

%-------------------------------
% Custom Commands
%-------------------------------
\newcommand{\hli}{\vspace{.25cm}\rule{\linewidth}{0.4pt}\vspace{.25cm}}
\newcommand{\m}{\vspace{1cm}}
\newcommand{\mm}{\vspace{0.01cm}}
\newcommand{\cm}{\vspace{-0.1cm}}
\def\andname{\hspace*{-0.5em}}
\renewcommand{\thefootnote}{\fnsymbol{footnote}}

\begin{document}
    \title{Intro to MBQC}
        \author{Liam Miller}
        \maketitle

    \onecolumngrid
        Computation is often introduced through concrete machines: logic gates, electronic circuits, and physical hardware.
        While this perspective is useful in practice, it can hide the underlying structure shared by all models of computation.
        At its core, computation is not about a particular device, but about how information is represented and transformed.

        In this text, we take a structural approach and view computation as the composition of maps or functions acting on well-defined spaces of information.
        This viewpoint allows classical and quantum computation to be discussed within a common mathematical language, with the main differences arising from the types of spaces involved and the transformations allowed on them.

        The goal of this text is to provide a clear and mathematically grounded introduction to measurement-based quantum computation (MBQC).
        The emphasis is on understanding structure rather than hardware, and on using precise definitions without unnecessary technical overhead.
        Along the way, we will connect MBQC to ideas from tensor networks and symmetry-based descriptions of quantum states, highlighting how familiar tools from linear algebra naturally lead to this model of computation.

        By approaching MBQC from this angle, we hope to make it feel less mysterious and more like a natural extension of ideas already present in classical and circuit-based quantum computation.\\

    \twocolumngrid

    \section{Classical Computation: A Structural View}

        Classical computation provides a useful starting point for developing a structural understanding of computation.
        Although the physical devices used to perform classical computations are familiar, the mathematical framework underlying them is both simple and powerful.
        By examining classical computation in abstract terms, we can isolate the essential ingredients common to all computational models: the representation of information, the transformations acting on that information, and the way these transformations are combined to form algorithms.

        In this section, we focus on these ingredients while deliberately setting aside implementation details.
        This allows us to describe classical computation in a way that will generalize naturally to quantum computation, where the same structural ideas reappear in a different mathematical setting.

        \subsection{Information Spaces}

            We begin by making the notion of information more precise.
            In a computational setting, information is represented by elements of a set, which we will call an \emph{information space}.
            Each element of this space corresponds to a possible configuration of the data relevant to a given task.

            In classical computation, information spaces are typically finite sets.
            For example, a single classical bit is described as an element of the set
            \[
            \mathcal{B} = \{0,1\},
            \]
            while a register of $n$ bits is described by the Cartesian product $\mathcal{B}^n$.
            More complicated data structures—such as strings, integers, or finite records—can be modeled using similar constructions.

            At this level, no assumptions are made about how information is physically stored or manipulated.
            The information space serves only as an abstract description of the possible states the system may occupy.

    \section{Quantum Computation: Linear Algebra Replaces Sets}
  
    \section{Measurement-Based Quantum Computing}
        \subsection{Resource States} 
            % Measurements as matrices, but what are we measuring?

\end{document}